% Proposal for Convex Latent Variable
% Author: 
%   Jimmy Lin
%   Ian Yen

\documentclass{article} % For LaTeX2e
\usepackage{nips14submit_e,times}
\usepackage{hyperref}
\usepackage{url}
\usepackage{amssymb,amsmath,amsfonts,latexsym,mathtext}
%\documentstyle[nips14submit_09,times,art10]{article} % For LaTeX 2.09


\title{Project Proposal: Convex Latent Variable}

%{{{ Authors 
\author{
Hippocampus\thanks{ Use footnote for providing further information
about author (webpage, alternative address)---\emph{not} for acknowledging
funding agencies.} \\
Department of Computer Science\\
Cranberry-Lemon University\\
Pittsburgh, PA 15213 \\
\texttt{hippo@cs.cranberry-lemon.edu} \\
\And
Coauthor \\
Affiliation \\
Address \\
\texttt{email} \\
\AND
Coauthor \\
Affiliation \\
Address \\
\texttt{email} \\
\And
Coauthor \\
Affiliation \\
Address \\
\texttt{email} \\
\And
Coauthor \\
Affiliation \\
Address \\
\texttt{email} \\
(if needed)\\
}
%}}}

% The \author macro works with any number of authors. There are two commands
% used to separate the names and addresses of multiple authors: \And and \AND.
%
% Using \And between authors leaves it to \LaTeX{} to determine where to break
% the lines. Using \AND forces a linebreak at that point. So, if \LaTeX{}
% puts 3 of 4 authors names on the first line, and the last on the second
% line, try using \AND instead of \And before the third author name.

\newcommand{\fix}{\marginpar{FIX}}
\newcommand{\new}{\marginpar{NEW}}

%{{{ Macros
\newcommand{\sumn}{\sum_{n}}
\newcommand{\sumk}{\sum_{k}}
\newcommand{\wnk}{w_{nk}}
\newcommand{\wn}{\mathbf{w}_n}
\newcommand{\wnbyn}{\mathbf{w}_{n\times n}}
\newcommand{\x}[1]{\mathbf{x}_{#1}}
\newcommand{\xn}{\mathbf{x}_n}
\newcommand{\muk}{\boldsymbol{\mu}_k} 
\newcommand{\LTwoNorm}[1]{||#1||^{2}}
\newcommand{\maxn}{ \underset{n}{\text{max}} }
%}}}

%\nipsfinalcopy % Uncomment for camera-ready version

%%% TODO list:
%% 1. reason of applying group-lasso (one or zero)
%% 2. 

\begin{document}

\maketitle

\begin{abstract}
    Add abstract here..
\end{abstract}

\section{Problem Formulation}
\subsection{Motivation}
Given a set of data entities $\x{1}, ..., \xn, ..., \x{N} $, we wish to
automatically cluster these entities without specifying the number of cluster
centroids. One possible approach is to regard these input entities as
potential cluster candidates and evaluate the "belonging matrix" $\wnbyn$, consisting
of latent variables $\wnk$ indicating the extent of one entity $\xn$ explained
by one potential cluster centroid candidate $\muk$.


\subsection{Overall Optimization Goal}
Formally, our goal is to achieve clustering with group-lasso regularization.
Hence, we formulate the target problem in terms of notations introduced above
as follows:  

 \begin{align}
 & \underset{x}{\text{minimize}}
 & & 
    J(x) = \frac{1}{2} \sumn \sumk \wnk \LTwoNorm{\xn - \muk}  % clustering loss
        + \lambda \sumk \maxn | \wnk |  \\ % group-lasso
 & \text{subject to} 
 & & 
 \forall n,\ \sumk \wnk \leq 1 \\
 & 
 & &
 \forall n,\ k,\ \wnk \geq 0
 \end{align}

Every entity must be assigned to some centroid with prob 1.
Every entity must have non-negative belonging to each centroid candidate.

Note that the group-lasso can either be one or zero in the context that we
pick up the most promising centroid candidates from provided entities. 

\subsection{Processing Techniques}


\subsubsection*{Acknowledgments}

\subsubsection*{References}
\small{

}

\end{document}
